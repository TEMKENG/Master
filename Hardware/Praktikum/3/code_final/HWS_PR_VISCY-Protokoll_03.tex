\documentclass{article}
\usepackage[T1]{fontenc}
\usepackage[ngerman]{babel}
\usepackage{beramono}% monospaced font with bold variant
\usepackage[pdftex]{graphicx}
\graphicspath{ {./images/} }
\usepackage{listings}
\usepackage{subcaption}
\usepackage{color}

\lstdefinelanguage{VHDL}{
	morekeywords={
		library,use,all,entity,is,port,in,out,end,architecture,of,
		begin,and
	},
	morecomment=[l]--
}

\lstdefinelanguage{ASSEMBLER}{
	morekeywords={
		add, sub, sal, sar, and:, or, xor, not, ld, st, jmp, halt, jz, jnz, org, data, data, end,start
	},
	morecomment=[l];
}
\usepackage{xcolor}
\colorlet{keyword}{blue!100!black!80}
\colorlet{comment}{green!90!black!90}
\lstdefinestyle{vhdl}{
	language     = VHDL,
	basicstyle   = \ttfamily,
	keywordstyle = \color{keyword}\bfseries,
	commentstyle = \color{comment}
}

\lstdefinestyle{assembler}{
	language     = ASSEMBLER,
	basicstyle   = \ttfamily,
	keywordstyle = \color{keyword}\bfseries,
	commentstyle = \color{comment}
}
\usepackage[margin=0.5in]{geometry}
\begin{document}
	\begin{center}
		
	\Huge
	Hardware-Systeme im SoSe 2020 Protokoll \\zum Praktikum	\\
	\vspace{1cm}
	\textbf{VISCY-Prozessor- Teil 3}
	\vspace{10cm}
	\begin{table*}[h!]
		\Huge
		\begin{tabular}{rl}
			Gruppenmitglieder: & \textbf{V}annessa \textbf{K}emeni \textbf{K}abiwo \\
							   &\textbf{T}hibaut \textbf{T}EMKENG \\
							   & \textbf{N}iko \textbf{M}aximiliam \textbf{M}azanec
		\end{tabular}
	\end{table*}
\end{center}
	
	\newpage
	\section{Vollständiges Steuerwerk.}
		\subsection{Zustandsübergangsdiagramm.}
			\begin{figure}[h!]
				\includegraphics[width=1\linewidth,height=0.5\textheight]{images/zustand}
				\caption{Zustandsdiagramm}
			\end{figure}
		\subsection{VHDL Code des Steuerwerks.}
		\footnotesize
		\lstinputlisting[style=vhdl]{controller.vhdl}	
		\subsection{Kommentierter Assembler Code von allen VISCY-CPU Befehlen.}
			\footnotesize
			\lstinputlisting[style=assembler]{test_all.asm}
		\subsection{Screenshot von Gtkwave.}
				\begin{figure}[h!]
				\includegraphics[width=\linewidth,height=0.3\textheight]{images/gtkwave_test_all}
				\caption{Gtkwave vom Befehlssatz}
			\end{figure}
		\normalsize
		Da die Gtkwave-Form nicht übersichtlich ist, haben wir für jede Durchführung von Befehlssatz die Ergebnis in den Speicher angelegt.Für jeder Befehlssatz haben wir eine Label in Assembler-Code gemacht.
		\subsection{Screenshot des Schaltbildes ('xsch')}
			\begin{figure}[h!]
				\includegraphics[width=\columnwidth,height=\textheight]{images/controller_synthese}
				\caption{Steuerwerk Synthese}
			\end{figure}
			\begin{table}[h!]
				\begin{tabular}{rl}
					Anzahl Flipflops:& 4\\
					maximale Verzögerungszeit: &Konsole:2802 ps XSCH: 2235 ps \\
					
				\end{tabular}
			\end{table}
	\section{IC-Layout}
		\subsection{Komplettes Chip}
				\begin{figure}[h!]
				\includegraphics[width=1\linewidth,height=0.8\textheight]{images/pc_chip_komplett}
				\caption{Chip Layout}
			\end{figure}
		\subsection{Standardzell-Bereich}
			\begin{figure}[h!]
			\includegraphics[width=1\linewidth,height=0.45\textheight]{images/nero_1}
			\includegraphics[width=1.\linewidth,height=0.45\textheight]{images/nero_2}
			\caption{Standardzell-Bereich}
		\end{figure}
		
	\section{Inbetriebnahme der PCU}
			\begin{figure}[h!]
				\includegraphics[width=1.\linewidth,height=0.5\textheight]{images/3-befehle}
				\caption{Die ersten drei Befehle}
			\end{figure}
			\vspace{1cm}
			\begin{figure}[h!]
				\includegraphics[width=1.\linewidth,height=0.5\textheight]{images/letzten1.PNG}
				\includegraphics[width=1.\linewidth,height=0.5\textheight]{images/letzten2.PNG}
				\caption{Der letzten drei Befehle (inklusive "halt")}
			\end{figure}
			\vspace{1cm}
			\begin{figure}[h!]
				\includegraphics[width=1.\linewidth,height=0.5\textheight]{images/laden1.PNG}
				\includegraphics[width=1.\linewidth,height=0.5\textheight]{images/laden2.PNG}
				\caption{Befehle, die die Quelloperanden aus dem Speicher lesen}
			\end{figure}
		\subsection{Statistik}




\end{document}